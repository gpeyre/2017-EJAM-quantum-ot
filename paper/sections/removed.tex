
%%% FIG %%%
\newcommand{\meshFig}[1]{\includegraphics[width=.105\linewidth]{mesh/interpol-#1}}
\newcommand{\ImgFig}[1]{\includegraphics[width=.105\linewidth]{2d/interpol-ellispses-#1}}
\begin{figure*}\centering
\begin{tabular}{@{\hspace{1mm}}c@{\hspace{1mm}}c@{\hspace{1mm}}c@{\hspace{1mm}}c@{\hspace{1mm}}c@{\hspace{1mm}}c@{\hspace{1mm}}c@{\hspace{1mm}}c@{\hspace{1mm}}c@{}}
\meshFig{1}&
\meshFig{1}&
\meshFig{2}&
\meshFig{3}&
\meshFig{4}&
\meshFig{5}&
\meshFig{6}&
\meshFig{7}&
\meshFig{7}\\
\ImgFig{1}&
\ImgFig{2}&
\ImgFig{3}&
\ImgFig{4}&
\ImgFig{5}&
\ImgFig{6}&
\ImgFig{7}&
\ImgFig{8}&
\ImgFig{9}\\
$t=0$ & $t=1/8$ & $t=1/4$ & $t=3/8$ & $t=1/2$ & $t=5/8$ & $t=3/4$ & $t=7/8$ & $t=1$ 
\end{tabular}
\caption{
Examples of interpolations obtained using formula~\eqref{eq-interpolating}. \gabriel{redo the ellipses}
%
\textbf{Top:} Interpolation on a 3-D surface (a triangulated mesh).
% 
The red ellipsoids depicts the tensors $\mu_t$ defined over the tangent planes and the coloring of the surface displays $\tr(\mu_t)$ (blue corresponding to $0$, yellow to large values).
\textbf{Bottom:} Interpolation on a 2-D planar domain, the background image is a texture synthesized from the underlying tensor field using an anisotropic diffusion applied to a Gaussian white noise initial condition.
} \label{fig:mesh}
\end{figure*}
%%% FIG %%%
