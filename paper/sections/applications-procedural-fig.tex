
%%% FIG %%%
\newcommand{\BumpFig}[1]{\includegraphics[width=.17\linewidth,trim=140 10 125 0,clip]{mesh-bump/anisodiffus-#1}}
\newcommand{\TextureImg}[2]{\includegraphics[width=.19\linewidth]{textures/#1/interpol-#2}}
\begin{figure}\centering
\begin{tabular}{@{}c@{\hspace{.5mm}}c@{\hspace{.5mm}}c@{\hspace{.5mm}}c@{\hspace{.5mm}}c@{\hspace{.5mm}}c@{}}
\TextureImg{2d-bump-donut}{render-1}&
\TextureImg{2d-bump-donut}{render-3}&
\TextureImg{2d-bump-donut}{render-5}&
\TextureImg{2d-bump-donut}{render-7}&
\TextureImg{2d-bump-donut}{render-9}&
\includegraphics[height=.19\linewidth]{textures/2d-bump-donut/colorbar.png} \\
\BumpFig{1}&
\BumpFig{3}&
\BumpFig{5}&
\BumpFig{7}&
\BumpFig{9}&\\
$t=0$ & $t=1/4$ & $t=1/2$ & $t=3/4$ & $t=1$
\end{tabular}
\caption{Example of interpolation between two input procedural anisotropic noise functions. The PSD tensor field parameterizing the texture are displayed on Figure~\ref{fig:intro}. The colormap used to render the anisotropic texture is displayed on the last column.  
} \label{fig:texture}
\end{figure}
%%% FIG %%%