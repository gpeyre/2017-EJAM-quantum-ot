% !TEX root = ../TensorOT.tex



\section{Applications}


%%%%%%%%%%%%%%%%%%%%%%%%%%%%%%%%%%%%%%%%%%%%%%%%%%%%%%%%%
\subsection{Anisotropic Texture Synthesis}

Texture synthesis using procedural noise functions are widely used in rendering pipelines and video games because of both their low storage cost and the fact that they are parameterized by a few meaningful parameters~\cite{LagaeSurvey}. Following~\cite{LagaImproving} we consider here a spatially-varying Gabor noise function (i.e. a non-stationary Gaussian noise), whose covariance function is parameterized using a PSD-valued field $\mu$. 
%
Quantum optimal transport allows to interpolate and navigate between these noise functions by transporting the corresponding tensor fields. 
%
The initial Gabor noise method makes use of sparse Gabor splattering~\cite{LagaeSurvey} (which enable synthesis at arbitrary resolution and zooming). We rather consider here a more straightfoward, where the texture $f_{t_0}$ is obtained by stopping at time $t=t_0$ an ansiotropic diffusion guided by the tensor field $\mu$  of a high frequency noise $\Nn$ (numerically a white noise on a grid)
\eq{
	\frac{\partial_t f_t}{\partial t} = \text{div}( \mu \nabla f_t ), \qwhereq
	f_{t=0} \sim \Nn, 
}
where $(\mu \nabla f_t)(x) \eqdef \mu(x) (\nabla f_t(x))$ is the vector field obtained by applying the tensor $\mu(x) \in \Ss_2^+$ to the gradient vector $\nabla f_t(x) \in \RR^2$. 
%
Locally around $x$, the texture is stretched in the direction of the main eigenvector of $\mu(x)$,  highly anisotropic tensor giving rise to elongated ``stripes'' as opposed to isotropic tensor generating ``spots''. 
%
Numerically, $f$ is discretized on a 2-D grid, and $\mu$ is represented on this grid as a sum of Dirac masses~\eqref{eq-input-measures}. 


%%
\textbf{Bottom row:} an example application is the generation of a spatially varying anisotropic texture that can be used as a bump map to offset the surface in the normal direction.



%%% FIG %%%
\newcommand{\BumpFig}[1]{\includegraphics[width=.17\linewidth,trim=140 10 125 0,clip]{mesh-bump/anisodiffus-#1}}
\newcommand{\TextureImg}[2]{\includegraphics[width=.19\linewidth]{textures/#1/interpol-#2}}
\begin{figure}\centering
\begin{tabular}{@{}c@{\hspace{.5mm}}c@{\hspace{.5mm}}c@{\hspace{.5mm}}c@{\hspace{.5mm}}c@{}}
\TextureImg{2d-bump-donut}{render-1}&
\TextureImg{2d-bump-donut}{render-3}&
\TextureImg{2d-bump-donut}{render-5}&
\TextureImg{2d-bump-donut}{render-7}&
\TextureImg{2d-bump-donut}{render-9}\\
\BumpFig{1}&
\BumpFig{3}&
\BumpFig{5}&
\BumpFig{7}&
\BumpFig{9}\\
$t=0$ & $t=1/4$ & $t=1/2$ & $t=3/4$ & $t=1$
\end{tabular}
\caption{Blabla. See \ref{fig:intro} for the ellipses.
} \label{fig:texture}
\end{figure}
%%% FIG %%%



%%%%%%%%%%%%%%%%%%%%%%%%%%%%%%%%%%%%%%%%%%%%%%%%%%%%%%%%%
\subsection{Anisotropic Material Modification}


SVBRDF~\cite{Aittala2015}


%%%%%%%%%%%%%%%%%%%%%%%%%%%%%%%%%%%%%%%%%%%%%%%%%%%%%%%%%
\subsection{Anisotropic Meshing}



%%% FIG %%%
\newcommand{\MeshingImg}[2]{\includegraphics[width=.195\linewidth]{meshing/#1/input-#2}}
\begin{figure}\centering
\begin{tabular}{@{}c@{}c@{}c@{}c@{}c@{}}
\MeshingImg{2d-bump-donut}{mesh-1}&
\MeshingImg{2d-bump-donut}{mesh-3}&
\MeshingImg{2d-bump-donut}{mesh-5}&
\MeshingImg{2d-bump-donut}{mesh-7}&
\MeshingImg{2d-bump-donut}{mesh-9}\\
\MeshingImg{images}{mesh-1}&
\MeshingImg{images}{mesh-3}&
\MeshingImg{images}{mesh-5}&
\MeshingImg{images}{mesh-7}&
\MeshingImg{images}{mesh-9}\\
$t=0$ & $t=1/4$ & $t=1/2$ & $t=3/4$ & $t=1$
\end{tabular}
\begin{tabular}{@{}c@{\hspace{1mm}}c@{\hspace{8mm}}c@{\hspace{1mm}}c@{}}
\MeshingImg{images}{images-1}&
\MeshingImg{images}{mesh-1-img}&
\MeshingImg{images}{images-2}&
\MeshingImg{images}{mesh-9-img}
\end{tabular}
\caption{Blabla.
} \label{fig:meshing}
\end{figure}
%%% FIG %%%




%%%%%%%%%%%%%%%%%%%%%%%%%%%%%%%%%%%%%%%%%%%%%%%%%%%%%%%%%
\subsection{Diffusion Tensor Imaging}


%%%%%%%%%%%%%%%%%%%%%%%%%%%%%%%%%%%%%%%%%%%%%%%%%%%%%%%%%
\subsection{Deformation Interpolation by Stress Transportation}

We propose here to navigate between deformations of a given solid by transporting the stress tensors field generated by these deformations. The stress tensor encodes the local amount and anisotropy of  stretch of the deformation. The proposed interpolation method is able to transport inside the shape regions of high compression/expansion.
%
While state of the art interpolation method use involved non-convex variational methods (typically modelling deformation as a manifold of diffeomorphisms\gabriel{cite LDDMM}), the proposed method is more ad-hoc but rely on a convex solver. 


From some spatial deformation $x \in \Om \rightarrow T(x)$ defined on some domain $\Om \subset \RR^d$, one considers the Jacobian $J(x) \eqdef \partial_x T(x)$
and performs its polar decomposition $J(x) = U(x)�\mu(x)$ where $U(x) \in \Oo_d$ is orthogonal and $\mu(x) \in \Ss_d^+$ is PSD. 
%
Here $\mu$ is the the so-called stress tensor field \gabriel{is it the correct wording from  continuum mechanics ?}, and it describes the amount, direction and anisotropy of stretch that a deformable body $\Om$ is subject to through the deformation $T$.
%
Note that it is important to to polar decompose $J(x)$ rather than $J(x)^\top$ because this way all the PSD matrices $\mu(x)$ are defined in a common reference frame.


The deformation $T$ is assumed to be a smooth diffeomorphism (so that the shape is not self-interpenetrating) and supposed without loss of generality to be orientation preserving. Then $\det(U(x))=1$ so that there is no ``singularity" in this decomposition, $U$ is a smooth ``correction" field of rotations \gabriel{Is it true?}.

\begin{rem}[Smaller deformations]
If $T$ is not a diffeomorphism (for instance if it arises from some optical flow computation), then one can consider a ``smaller'' deformation $T^\de(x) \eqdef (1-\de) x +  \de (T(x)-x)$  so that for $\de$ small enough $T^\de$ becomes a diffeomorphism.
\end{rem}


\if 0
\begin{rem}[Small deformation asymptotic]
Note that as $\de \rightarrow 0$, one obtains the linearized decomposition of $J^\de(x) \eqdef \partial_x T^\de(x)$ as  $T^\de(x) = U^\de(x) \mu^\de(x)$ using
\begin{align*}
  U^\de(x) &= \Id_{d \times d} + \de D(x) + O(\de^2), \\
  \mu_t(x) &= \Id_{d \times d} + \de C(x) + O(\de^2)
\end{align*}
where $C(x) \eqdef (J(x)+J(x)^\top)/2-\Id_{d \times d}$ is the Cauchy stress tensor and $D(x) = (J(x)-J(x)^\top)/2$ is a deviator tensor anti-symetric part. This shows that typically $\de$ should be chosen of the order of $-1/\min_x(\text{eig}(C(x)))$ to obtain a proper diffeomorphism.
\end{rem}

\begin{rem}[Anisotropy boosting]
This parameter $\de$ controls the anisotropy of $\mu^\de$. This means that even if the input $T$ is a diffomorphism, one can ``boost" it to increase the anisotropy of $\mu$ by choosing $\de>1$. This is an extra parameter of the method to rescale the tensors.
\end{rem}
\fi

\gabriel{Maybe use $(T_0,T_1)$ in place of $(T,S)$. }
Given two input deformation field $T(x)$ and $S(x)$, we denote their Jacobian's decomposition $(J(x)=U(x) \mu(x),K(x)=V(x)\nu(x))$. We use quantum OT to interpolate between these deformation by interpolating the Jacobians. 

\gabriel{Explain sampling and discretization. }
We first solve the OT~\eqref{eq-Kantorovich} to obtain the coupling $\ga$ (using Sinkhorn's iterations) between $\mu$ and $\nu$. We then transport the rotation fields $(U,V)$ fields along the coupling. We simply update the formula~\eqref{eq-interpolating} to account also for the rotational part, and define
\eql{\label{eq-interpolating}
	\foralls t \in [0,1], \quad
	J_t \eqdef \sum_{i,j} \ga_{i,j}^t \de_{x_{i,j}^t}.
}
where now we define 
\eq{
	\ga_{i,j}^t \eqdef [(1-t) \bar\muA_i + t \bar\muB_{j}] \ga_{i,j}.
}
\eq{
	\bar\muA_i = J_i \Big( \sum_{j} \ga_{i,j} \Big)^{-1} 
	\qandq
	\bar\muB_j = K_j \Big( \sum_{i} \ga_{i,j} \Big)^{-1}
}
One reconstructs using a least square the deformation
\eq{
	T_t \eqdef \uargmin{T} \norm{\partial_x T-J_t}.
}
Note that actually this minimisation needs to be corrected to account for the kernel of $\partial_x$ \gabriel{I guess fixing the mean or a point is enough}. 
% where $\tilde T_t = (1-t) T+t S$ is for the linear interpolation (but other reference can be used).

This interpolation satisfies $T_{t=0}=T$ and $T_{t=1}=S$, and it should take into account compressions/dilatations/stretches occurring are different places in the shape. 


%%%%%%%%%%%%%%%%%%%%%%%%%%%%%%%%%%%%%%%%%%%%
\subsection{Shape Registration}

Only use $W$ as a fidelity term in diffeomorphic registration when representing shapes as tensor measures. 

We denote $\mu$ the input measure to register to a target measure $\nu$, which are defined on $\RR^d$ and takes values in $\Ss_d^+$. 
%
The action of a diffeomorphism $T : \RR^d \rightarrow \RR^d$ on $\mu$ is defined as
\eq{
	T_\sharp \mu \eqdef \sum_i [T'(x_i)]^{-1} \mu_i [T'(x_i)]^{-1,\top} \de_{T(x_i)}.
}
This action of warpings on tensor-valued measures is more involved than for the case of scalar valued measures (as done for instance in~\cite{}). Indeed, this action not only displaces the input masses but also rotate and rescale the tensors by the conjugation by the Jacobian $T(x_i)^{-1} \in \RR^{d \times d}$. \gabriel{Show an example} 

Assuming a parametrization $\th \mapsto T_\th$ of the deformation (one could of course use more involved non-parametric model, for instance~\cite{daemon,lddmm}), the registration problem corresponds to solving 
\eq{
	\min_\th \Ee(\th) = W(T_{\th,\sharp} \mu, \nu), 
}
where the position $x(\th)$ are typically parameterized using a diffeomorphic model. 

Using the chain-rule, \gabriel{this formula is actually wrong because it lacks the differential of the conjugation term in $T_{\th,\sharp}$, which seems intractable to compute.}
\eq{
	\nabla \Ee(\th) = [\partial_\th T_\th(x)]^* [\nabla_x W(T_{\th,\sharp} \mu, \nu)], 
}
where $[\partial_\th T_\th(x)]^*$ is the adjoint of the differential of $\th \mapsto (T_\th(x_i))_i$.
%where typically the adjoint Jacobian $[\partial T(\th)]^*$ is implemented using a adjoint state method or reverse automatic�code differentiation.
%
Assuming for simplicity $c_{i,j} = \norm{x_i-y_j}^2\Id_{d \times d}$, the gradient of $x \mapsto  W(\mu,\nu)$ (where $\mu$ is parameterized by $x=(x_i)_i$) is given by 
\eq{
	\nabla_x W(\mu,\nu) = (2 \sum_j \tr(\ga_{i,j}) (x_i-y_j))_i
}
where $\ga$ is a solution of~\eqref{eq-kantorovitch-regul} computed using Sinkhorn iterations.